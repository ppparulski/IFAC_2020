%===============================================================================
% $Id: ifacconf.tex 19 2011-10-27 09:32:13Z jpuente $  
% Template for IFAC meeting papers
% Copyright (c) 2007-2008 International Federation of Automatic Control
%===============================================================================
\documentclass{ifacconf}

%\usepackage{graphicx}      % include this line if your document contains figures
\usepackage{natbib}        % required for bibliography

\usepackage[pdftex]{graphicx}
   \graphicspath{{./img/}}
   

\usepackage[usenames,dvipsnames]{xcolor}

\usepackage{amsmath} % assumes amsmath package installed
\usepackage{amssymb}  % assumes amsmath package installed
%\usepackage{program}
\usepackage{algorithm,algpseudocode}
%\usepackage{algpseudocode}

%---------------0---------------------
%---------------0---------------------
%---------------0---------------------
% USUNAC TE PAKIETY NA KONCU

%\usepackage{polski}
\usepackage[cp1250]{inputenc}
\usepackage[OT4]{fontenc}
%---------------0---------------------
%---------------0---------------------
%---------------0---------------------
%---------------0---------------------




%===============================================================================
\begin{document}
\begin{frontmatter}

%\title{Style for IFAC Conferences \& Symposia: Use Title Case for
%  Paper Title\thanksref{footnoteinfo}} 
\title{A hybrid controller for the Pendubot based on linearization methods} 
% Title, preferably not more than 10 words.

%\thanks[footnoteinfo]{Sponsor and financial support acknowledgment
%goes here. Paper titles should be written in uppercase and lowercase
%letters, not all uppercase.}


\author[First]{Pawe\l{} Parulski} 
\author[First]{Patryk Bartkowiak} 
\author[First]{Dariusz Pazderski}
\author[First]{Krzysztof Koz\l{}owski}

\address[First]{Institute of Automation and Robotics, 
	Poznan University of Technology, Poznan, Poland 
	(e-mail: pawel.parulski@put.poznan.pl, patryk.bartkowiak@put.poznan.pl, dariusz.pazderski@put.poznan.pl, krzysztof.kozlowski@put.poznan.pl).}



\begin{abstract}                % Abstract of not more than 250 words.
The aim of this paper is to
test the usefulness of a new approach based on  partial feedback linearization to control the Pendubot.
%
The control problem stated in the article is to stabilize the Pendubot in the upright position.
\end{abstract}

\begin{keyword}
Robotics, Robot dynamics, Pendubot, nonlinear control, feedback linearization, nonlinearity 
%Five to ten keywords, preferably chosen from the IFAC keyword list.
\end{keyword}

\end{frontmatter}
%===============================================================================

\section{Introduction}

Acrobot and Pendubot are well-known fundamental examples of underactuated  mechanical systems moving in the presence of gravity. Although their kinematic structures are relatively simple, still they can be considered as important benchmark systems which are of great importance for the development of nonlinear control strategies.  

Basically, a standard approach to stabilize both systems at an unstable equilibrium requires design of two controllers for the task of swing-up (when the robot hangs down) and the balancing task around the equilibrium point, respectively.
%
Many solutions to the first problem are based on the work of \cite{Spong_partial, xin_swing_new}, the second, is still being developed by many authors.

Recently, in \cite{Respondek_2019} a new insight to control of underactuated systems is investigated. The presented approach is based on finding the largest feedback linearizable subsystem for mechanical systems that are not fully feedback linearizable. Namely, the authors are focused on the problem of choosing a partially linearizing output that renders the zero dynamics asymptotically stable,
and illustrate the theoretical results by mathematical formulas developed for double inverted pendulum with one actuator (Acrobot and Pendubot). Nonetheless, no attempt is done to apply the discussed concept for a constructive design of a controller for these benchmark systems. Consequently, any simulation or experimental verification has not been reported yet.
%
In this paper, however, authors attempt to fill this gap and extend preliminary results recently discussed by \cite{bartkowiak} for the Pendubot. Authors are also interested in a possible application of such an algorithm for the physically existing robot.
%

{\color{black}An important issue of linearization approach using state and feedback transformations discussed in \cite{Respondek_2019} is singular points. Hence, a feasible state space is constrained and a feedback cannot be designed in a global way. In the case of Pendubot, a severe constraint comes from the requirement that the system should be in a moving phase. In order to overcome these limitations we propose a hybrid controller which is supported by a simple linear feedback. Then we consider the performance of closed-loop system based on numerical analysis. In particular we search for admissible states of the Pendubot for which the hybrid algorithm ensures the convergence to the equilibrium point and makes this point be asymptotically stable.}

The paper is organized as follows. Section \ref{sec:modelowanie} describes the mathematical model for the Pendubot system, together with constrains imposed on the model. 
%
In section \ref{sec:stabilization_problem} the analyzed control algorithms along with the state transformation for the stabilization task is introduced.
%
Section \ref{sec:symulacje_i_porowanie} describes the simulation results of analysed algorithms, while the Section \ref{sec:konkluzje} gives final remarks.



% % % % % % % % % % % % % % % % % % % % % % % % % % % % % %


\section{Model}
\label{sec:modelowanie}

The Pendubot is a connection of  $N = 2$ rigid bodies coupled  in a tree structure, supported on ground via an actuated frictionless revolute joint. Both links have non-zero mass and the revolute joint connecting them is unactuated. As a result, the system has one degree of underactuation ($2$ DOF with $1$ independent actuator).
%
\begin{figure}[!ht]
\centering
\includegraphics[height=0.3\textwidth]{2_link}
\caption{Two link inverted pendulum, with one actuator -- Pendubot}
\label{fig:rysunek_1}
\end{figure}
%e is depicted. 
%
In Fig. \ref{fig:rysunek_1} a standard pendulum structure is depicted. The reference frame is attached {at the pivot point}, and coordinates are indicated as $\theta = \left[\theta_1\ \theta_2\right]^\top\in S^1\times S^1$. In order to establish system dynamics one can define Lagrangian $L = K - V$, while  $K = \frac{1}{2} \dot{\theta}^T D(\theta) \dot{\theta}$ denotes kinetic energy, with $D$ being a positive definite inertia matrix, and $V$ is the potential energy. Next, taking into account the actuation on the system one obtains 
%
\begin{equation}
\frac{d}{dt}\frac{\partial L}{\partial\dot{\theta}_{k}}-\frac{\partial L}{\partial \theta_{k}}=\begin{cases}
\tau_{k}, & k=1\\
0, & k=2
\end{cases}
\label{eq:wzor_1}
\end{equation}
%
with $\tau_k\in\mathbb{R}$.
%
The mathematical model of the system dynamics thus takes the following standard form
%
\begin{equation}
D(\theta) \ddot{\theta} + C(\theta, \dot{\theta}) \dot{\theta} + G(\theta) = B \tau,
\label{eq:wzor_2}
\end{equation}
%
where corresponding entries of $D(\theta)\in\mathbb{R}^{2\times 2}$ satisfy
%
\begin{equation}\label{eq:dynamika_wzorki_na_d_ii}
\begin{aligned}
d_{11}(\theta_2) =&  a_{1} + a_2 + 2a_{3}  \cos \theta_2,\\
d_{12}(\theta_2) =&  a_{2} + a_3  \cos \theta_2,\\
d_{21}(\theta_2) =&  d_{12},\\
d_{22}(\theta_2) =&  a_{2},
\end{aligned}
\end{equation}
%
while
$
a_1 = m_1 L_{c_1}^2 + m_2 L_1^2 + I_1
$, 
$
a_2 = m_2 L_{c_2}^2 + I_2
$, 
$
a_3 = m_2 L_1 L_{c_2}
$, 
$
a_4 = m_1 L_{c_1} + m_2 L_1
$, 
$
a_5 = m_2 L_{c_2}
$, 
%\\
%
\begin{equation}
C(\theta,\dot{\theta}) = 
\begin{bmatrix}
-2 a_3 \sin\theta_2 \dot{\theta_2} & -a_3 \sin\theta_2(\dot{\theta}_1 + \dot{\theta_2})\\
a_3 \sin\theta_2 \dot{\theta}_1 & 0
\end{bmatrix}\in\mathbb{R}^{2\times 2},
\label{eq:dynamika_wzorek_C}
\end{equation}
%
%
%
\begin{equation}
G(\theta) = 
\begin{bmatrix}
g a_5 \cos(\theta_1 + \theta_2) + a_4 \cos\theta_1 
\\
g a_5 \cos(\theta_1 + \theta_2)
\end{bmatrix}\in\mathbb{R}^{2},
\label{eq:dynamika_wzorek_G}
\end{equation}
%
$B=[1 \;\; 0\,]^\top\in\mathbb{R}^{2}$ and $\tau\in\mathbb{R}$ is the control input.



%%%%%%%%%%%%%%%%%%%%%%%%%%%%%%%%%%%%%%%%%%%%%%%%%%%%%%%%%


\section{Stabilization problem}
\label{sec:stabilization_problem}


{\color{black}Taking into consideration the last terms in Eqs. \eqref{eq:dynamika_wzorek_G} and 
\eqref{eq:wzor_1}}, one can obtain the following relationship describing the equilibrium condition:
$
\cos(\theta_1 + \theta_2) = 0
$.
%
However, for $\tau_1=0$ system (\ref{eq:wzor_1})  has four characteristic equilibrium points: 
first of them are two unstable positions (top and mid position) 
where 
$
(q_1, q_2, \dot{q}_1, \dot{q}_2) = (\frac{\pi}{2}, 0,0,0)
$
and 
$
(q_1, q_2, \dot{q}_1, \dot{q}_2) = (-\frac{\pi}{2}, \pi,0,0)
$, 
respectively.
The other two, determined by  
$
(\frac{\pi}{2}, \pi,0,0)
$ 
and
$
(-\frac{\pi}{2}, 0,0,0),
$ 
are trivial and they are not being analyzed here.

The aim of the work is to examine the usability of a hybrid controller using the formalism presented in \cite{Respondek_2019} to stabilize a Pendubot around its top unstable position, taking into account the limitations and constraints resulting from practical or physical conditions (existing robot).

\subsection{Control algorithm}
\label{sec:respondek}
%
The application of method discussed in \cite{Respondek_2019} to control the Pendubot can be briefly characterized  as follows.
At first, rewrite Eq. (\ref{eq:wzor_2}) in the following form:
%
\begin{equation}
\label{eq:Respondek_Acrobot_dynamika_2}
\begin{array}{lcl}
d_{11}  \ddot {\theta}_1 + d_{12}  \ddot {\theta}_2 + \mu_1 + \phi_1 &=& \tau
\\
d_{21}  \ddot {\theta}_1 + d_{22}  \ddot {\theta}_2 + \mu_2 + \phi_2 &=& 0
\end{array}
\end{equation}
%
where $\mu_1 = C_{11}(\theta, \dot \theta) \dot \theta_1 + C_{12}(\theta, \dot \theta) \dot \theta_2$, 
$\mu_2 = C_{21}(\theta, \dot \theta) \dot \theta_1$,
$\phi_1 = G_1(\theta)$, 
$\phi_2 = G_2(\theta)$, 
where $C_{ij}$, $G_i$ are entries of  Eq.~(\ref{eq:dynamika_wzorek_C}) and Eq.~(\ref{eq:dynamika_wzorek_G}), respectively, 
%
and  apply the feedback transformation according to \cite{Spong_partial}.
%\noindent
The overall robot dynamics model after this transformation can be represented by
%Calkowity model dynamiki Acrobota po transformacji mozemy zapisac w nastepujacej postaci:
%
%
\begin{equation}
\label{eq:respondek_forma_spong}
\Sigma_{pend}: \qquad
\begin{array}{rcl}
\dot{\theta}_1 &=& w_1
\\
\dot{w_1} &=& u
\\
\dot{\theta}_2  &=& w_2
\\
\dot{w_2} &=& -d_{22}^{-1} \mu_2 - d_{22}^{-1}\phi_2 + J_2(\theta_2) u.
\end{array}
\end{equation}
%
%
where:
$
J_2 (\theta_2) = - d_{22}^{-1} d_{21}. 
$
%
After introducing {\color{black}the following new state variables}
%
\begin{equation}
\label{eq:Respondek_nowe_zmienne}
\begin{array}{rcl}
q_1 &=& \theta_1 - I_2 (\theta_2)
\\
v_1 &=& w_1 - J_2^{-1} (\theta_2) w_2
\\
q_2 &=& \theta_1
\\
v_2 &=& w_1,
\end{array}
\end{equation}
%
%
where
%
$
I_2 (\theta_2) = \int_{0}^{\theta_2} J_2^{-1}(s) ds
$,
%and 
and {\color{black}transforming \eqref{eq:respondek_forma_spong} into normal form, one obtains equivalent dynamics defined by}
%
%
%model dynamiki Acrobota przybiera postac
%
\begin{equation}
\label{eq:Respondek_pendubot_nowe}
	\begin{array}{rcl}
\dot{q}_1 &=& {v}_1 
\\
\dot{{v}}_1 &=&  \alpha v_1^2 + \beta v_1 v_2 + \gamma v_2^2 + \eta
\\
\dot{q}_2 &=& v_2
\\
\dot{v}_2 &=& u,
\end{array}
\end{equation}
%
where $
\alpha (\theta_2)
=
\frac{a_3 \sin \theta_2}{a_2}
$, 
%
$
\beta (\theta_2)
=
- \frac{a_3 \sin \theta_2}{2 a_2}
$, 
%
$
\gamma (\theta_2)
=
\frac{a^2_3 \sin \theta_2 \cos \theta_2}{a_2 (a_2 + a_3 \cos \theta_2)} 
$, 
%
$
\eta (\theta_1, \theta_2)
= 
\frac{a_5 \cos (\theta_1+\theta_2)}{a_2 + a_3 \cos \theta_2}
%
$.
%
Correspondingly, Eq.~\eqref{eq:Respondek_pendubot_nowe} can be written as follows
%
\begin{equation}
\dot{x} = f(x) + g(x) u,
\label{eq:dynamika_f_x_g}
\end{equation}
%
where {\color{black}$x = [q_1\ {v}_1\ q_2\ v_2]^\top$ is a new state},
%
\begin{equation}
f(x) =
\left[
\begin{array}{c}
{v}_1  
\\
\alpha v_1^2 + \beta v_1 v_2 + \gamma v_2^2 + \eta
\\
v_2
\\
0
\end{array}
\right]
,
%
\quad
%
g(x) =
\left[
\begin{array}{c}
0\\0\\0\\1
\end{array}
\right].
\label{eq:wektory_f_oraz_g}
\end{equation}

{\color{black}Now in order to find control $u$ we look for such an output function $h$ that maximally linearizes \eqref{eq:Respondek_pendubot_nowe}.}
According to \cite{Respondek_2019}, the output function can be selected as:
% 
%
\begin{equation}
\label{eq:respondek_f_wyjscia_h_pendubot}
{\color{black}h} =  q_1 - q_{r1},
\end{equation}
%
where $q_{r1}\in S^1$ denotes the desired coordinate.
%
{\color{black}Then the following stabilizing feedback can be proposed}
%
\begin{equation}
u = u_h = \frac{1}{L_g L_f^2 h} (-L_f^3 h + X),
\label{eq:RE_acrobot_u}
\end{equation}
%
{\color{black}where $L_f h$, $L_f^2 h$ are Lie derivatives along vector field $f(x)$ and}
%
\begin{equation}
X = -(\omega_0^3 h + 3\omega_0^2 L_f h + 3\omega_0 L_f^2 h),
\label{eq:RE_acrobot_u_X}
\end{equation}
%
{\color{black}with $\omega_0>0$ being a gain scaling factor.}

{\color{black}According to \cite{Respondek_2019} relative degree of $h$ is not well defined around $v_1=v_2=0$, thus the equilibria of Eq.~(\ref{eq:dynamika_f_x_g}) cannot be regular.}
%
Since at {\color{black} non regular} points $L_g L_f^2 h=0$, from \eqref{eq:RE_acrobot_u} one can easily conclude that control signal $u$ explodes to plus or minus infinity. 
%
This means that the Pendubot cannot be part-linearizable, with a 3-dimensional linear subsystem around an equilibrium point.
%
As a result, one needs to use a different control algorithm around equilibrium, to settle the robot state at required upright position.
%

{\color{black}
Taking into account these limitations, the control strategy is proposed as follows.
Firstly, a set of feasible initial conditions for which the controller~(\ref{eq:RE_acrobot_u}) applied for system~(\ref{eq:dynamika_f_x_g}) brings the robot state near the equilibrium point should be obtained.
%(Tutaj mo�e rysunek jak si� uda go zrobi�).
\\
Secondly, the convergence area  for system~(\ref{eq:dynamika_f_x_g}) with the use of the following linear controller %(\ref{eq:sterownik_liniowy})
%
\begin{equation}
u_{Lin} = -K (x_r - x),
\label{eq:sterownik_liniowy}
\end{equation}
%
which stabilizes the robot in the equilibrium pose, should be determined. {\color{black}In \eqref{eq:sterownik_liniowy} terms $x_r = [q_{r1}\ v_{r1}\ q_{r2}\ v_{r2}]^\top$ and $K = [k_1\ k_2\ k_3\ k_4]$ stand for the reference state and the controller gains, respectively.}
% (Fig.~\ref{fig:obszar_zbierznosci_tylko_liniowy}).
%%
%\begin{figure}[!ht]
%	\centering
%	\includegraphics[width=0.85\linewidth]{img/RE_pendu_wykres_zbieznosci_bez_sat_lqr}
%	\caption{The convergence area  for the linear controller}
%	\label{fig:obszar_zbierznosci_tylko_liniowy}
%\end{figure}
%

{\color{black}As both controllers have a different convergence area, one can propose to combine them in order to increase this area, obtaining the control law for the Pendubot  in the following form}
%Jako, �e oba sterowniki maj� inny obszar zbie�no�ci, postanowiono dokona� ich po��czenia w celu zwi�kszenia tego� obszaru, uzyskuj�c zale�no�� na sterowanie dan� wzorem 
%
\begin{equation}
\label{eq:pendubot_u_hybrydowe}
u = 
\begin{cases}
%\frac{1}{L_g L_f^2 h} (-L_f^3 h + X), 
u_h 
& \text{for  
{\color{black}swing},}
\\
%-K (x_r - x), 
u_{Lin}
& \text{for {\color{black}stabilisation}}.
\end{cases}
\end{equation}
%
{\color{black}Consequently, the proposed control law for the Pendubot consists of two regulators.} The first one is a swing-up-like type, Eq.~\eqref{eq:RE_acrobot_u}, whose task is to bring the manipulator near the equilibrium point.
When the robot state is in a set containing the equilibrium it switches to the linear controller, Eq.~\eqref{eq:sterownik_liniowy},
%
which is designed to keep the system at the equilibrium pose.
%




\subsection{Robot parameters}
\label{sec:parametry_robota}

The choice of the robot parameters (Table~\ref{tab:parametry_robota}) in the simulation  was selected in order to adapt them to the physically existing mechanism, i.e. to the one-legged robot presented in Fig.~\ref{fig:sprzet}.
This one-legged mechanism can be studied as a double inverted pendulum.
Considered robot was built in the Institute of Automation and Robotics at  Poznan University of Technology, as a testbed for a four-legged walking robot presented in \cite{MKP:2008, parulski_michalski}.
%The robot parameters are collected in Table \ref{tab:parametry_robota}.
%
%
\begin{table}[!ht]
	\caption{Robot parameters}
	\label{tab:parametry_robota}
	\centering
	\begin{tabular}{|c|c|c|c|c|}
		\hline 
		Link & Mass & Centre of mass & Length & Inertia
		\\ 
		%\hline 
		&  [$\mathrm{kg}$] & [$\mathrm{m}$] & [$\mathrm{m}$] & [$\mathrm{kg\, m^2}$]
		\\ 
		\hline 
		1 & 1.593 & 0.074 & 0.15 & 0.0119
		\\ 
		\hline 
		2 & 0.405 & 0.134 & 0.295 & 0.0117
		\\ 
		\hline 
		
	\end{tabular} 
\end{table}
%
%
%
The physical model depicted in Fig.~\ref{fig:sprzet} is driven by Maxon 200W EC-Powermax~30 brushless motors with planetary gearhead of $n=53$ reduction.
Such drives provides the maximum torque of approximately $6$~Nm.
This value is considered as a saturation of the control input.
% which defines an additional constraint imposed on the model.
%
%
\begin{figure}[!ht]
	\centering
	\includegraphics[width=0.85\linewidth]{img/sprzet}
	\caption{Experimental test-bed}
	\label{fig:sprzet}
\end{figure}





\subsubsection{Tuning the controllers parameters}



To acquire the best parameter  $\omega_0$, the integral of time-weighted absolute error (ITAE, Eq. (\ref{eq:ITAE})) performance index was used [\cite{tuning_itae}]
%
\begin{equation}
ITAE = \int_{0}^{\infty} t |e(t)| dt,
\label{eq:ITAE}
\end{equation}
%
where $t$ is  the  time  and $e(t)$ is  the  difference between reference value and controlled variable.
%
The best index value coincided with the $\omega_0 = 11$.
%
In the following step the $K$-vector was obtained, which is the linear controller's gain.
In order to calculate them, the following cost function $J$, dependent on state and control signal, is defined
%
%
%
\begin{equation}
J = \int_t^\infty \left(x^T Q x + u^T R u\right) d\tau,
\label{eq:funkcja_J_lqr}
\end{equation}
%
%gdzie Q, R s� macierzami wag dla ka�dego parametru.
where $Q$ and $R$ are the weight matrices for states and inputs, respectively.
The optimal $K$-gains ($K = [1, 520, 1050, 50]$) were obtained for the linear approximation of Eq.~(\ref{eq:wzor_2}) at the equilibrium point with $Q = 1000 I_{4\times 4}$ and $R = 0.5$.
%





% % % % % % % % % % % % % % % % % % % % % % % % % % % % % % % % % % %


\section{Simulation results}
\label{sec:symulacje_i_porowanie}




This section provides simulation results of implementing control method \eqref{eq:pendubot_u_hybrydowe} to the system in a form of \eqref{eq:dynamika_f_x_g}.
{\color{black}The aim of presented simulations is to check the performance of the controller applied for the stabilization of Pendubot in the upright position. Another purpose is to verify the convergence area for the closed-loop system,} which means that the set of initial conditions are going to be checked to see whether they are appropriate to stabilize the robot under certain boundary conditions.
%
The algorithm verification is carried out in two ways.
Firstly, one wants to know how the algorithm behaves when no restrictions on the motor torque are imposed (Case I).
Secondly, the motor torque saturation is imposed on the model (Case II). {\color{black}Taking into account properties of the real robot  the torque magnitude is restricted to $6$~Nm.}
Thus, the algorithm is being checked whether is capable of controlling the model of physical robot.
%

It is interesting to see how far the Pendubot can be moved away from the equilibrium pose, to be stabilized in the upright position under certain boundary conditions. In simulations it was assumed that the desired stabilization pose is the upright position for which the angles $\theta_{1r}$ and $\theta_{2r}$ were equal $\frac{\pi}{2}$ and $0$, respectively.
%

To determine the convergence area of analyzed method, a~series of tests were conducted.
To check {\color{black}properties of the algorithm near} the equilibrium pose, the system  (\ref{eq:dynamika_f_x_g})  with only linear controller (\ref{eq:sterownik_liniowy}) working is checked.

As a~result one obtained a~region of initial conditions (Fig.~\ref{fig:obszar_zbierznosci_tylko_liniowy}) that leads to upright position.
%
\begin{figure}[!ht]
	\centering
	\includegraphics[width=0.85\linewidth]{img/RE_pendu_wykres_zbieznosci_bez_sat_lqr}
	\caption{A~region of initial conditions (blue colour) that leads to upright position for linear controller.}
	\label{fig:obszar_zbierznosci_tylko_liniowy}
\end{figure}
%
The plot of angular positions in links, ground reaction forces, {\color{black}the determinant of the decoupling matrix (term $M_\delta = L_g L_f^2 h$)} and motor torque for exemplary successful trial are depicted in Figs.~\ref{fig:LQR_pendu_th1th2_reakcja},~\ref{fig:LQR_pendu_mianownik_tau}.

Subsequently, the hybrid controller (\ref{eq:pendubot_u_hybrydowe}) was considered to verify {\color{black}the effectiveness of the proposed} control scheme.
%
\begin{figure}[!h]
	\centering
	\includegraphics[width=0.49\linewidth]{img/pendubot_th1th2_z_lqr}
	\includegraphics[width=0.49\linewidth]{img/Fr_bez_sat}
	\caption{The result of LQR controller -- a) Angular positions of links, b) Ground reaction forces}
	\label{fig:LQR_pendu_th1th2_reakcja}
\end{figure}
%
%
\begin{figure}[!h]
	\centering
	\includegraphics[width=0.49\linewidth]{img/M_d_z_lqr}
	\includegraphics[width=0.49\linewidth]{img/tau_z_lqr}
	\caption{The result of LQR controller -- a) The determinant of decoupling matrix $M_\delta = L_g L_f^2 h$, b) Motor torque}
	\label{fig:LQR_pendu_mianownik_tau}
\end{figure}
% 
{\color{black}To determine which control law should be chosen during the stabilization process, the following switching procedure was implemented. One can present this procedure as a~pseudo-code described by Algorithm~\ref{code:alg_1}:
%
\begin{algorithm}
	\caption{Switching procedure for Eq.~(\ref{eq:pendubot_u_hybrydowe})}
	\label{code:alg_1}
	\begin{algorithmic}[1]
%		\State $distance\gets  0 $
%		\State $time\gets  0 $
%		\State $speedGPS  \gets  0 $
		\If{
			$(\theta_{2_0} > a_1 \cdotp \theta_{1_0} + b_1)$	
			AND $(\theta_{2_0} < a_2 \cdotp \theta_{1_0} + b_2)$
			OR $(\left|\gamma\right| < \gamma_d)$
%			\OR {$\gamma < \gamma_{ref}$}
		}
		\State $u = u_{Lin}$
		\ElsIf
%		{$\frac{1}{L_g L_f^2 h} > 0.001$}
		{$\left| L_g L_f^2 h\right| > \delta$}
		\State $u = u_h$
		\Else
		\State $u=0$
		\EndIf
	\end{algorithmic}
\end{algorithm}
%
where $a_1, a_2, b_1, b_2$ are coefficients of straight  lines $y_i = a_i \cdotp \theta_{j,0} + b_i$ for $i=\{1,2\}$ which stand for the lower and upper bound of a convergence area illustrated in Fig.~\ref{fig:obszar_zbierznosci_tylko_liniowy}, 
and $\gamma_d$ is \color{black} the chosen upper bound of $|\gamma|$} from Eq~\eqref{eq:wektory_f_oraz_g}.
}
%

%The algorithm is not able to stabilize the Pendubot from any initial conditions, with the adopted assumptions.

For the Case~I and Case~II the results are depicted in Fig.~\ref{fig:obszar_zbierznosci_stero_hybrid_bez_saturacji} and  in Fig.~\ref{fig:obszar_zbierznosci_stero_hybrid_z_saturacja}, respectively.  
%
The blue cell indicates the successful trial, while the red one indicates the initial conditions that lead to failure (robot collapse).
%
%
%
\begin{figure}[!ht]
	\centering
	\includegraphics[width=0.85\linewidth]{img/RE_pendu_HYBRID_wykres_zbieznosci_bez_sat}
	\caption{Convergence area -- Case I (blue  -- successful trial, red -- failure)}
	\label{fig:obszar_zbierznosci_stero_hybrid_bez_saturacji}
\end{figure}
%%
\begin{figure}[!ht]
	\centering
	\includegraphics[width=0.85\linewidth]{img/RE_pendu_HYBRID_wykres_zbieznosci_z_sat}
	\caption{Convergence area -- Case II}
	\label{fig:obszar_zbierznosci_stero_hybrid_z_saturacja}
\end{figure}
%
%
Simulations  were conducted with a step size equal to 5$^\circ$, within range of $\theta_{1_0}$ from $0$ to $180^\circ$ and
$\theta_{2_0}$ from $-90^\circ$ to $90^\circ$, which gave more than 1300 simulation per approach.
These initial conditions mean that the swing-up problem is not considered, i.e. only  scenarios where the robot is above the ground are taken into account.
The trial was being taken as successful while the error between actual and desired angle was smaller than  $1^\circ$ within simulation time  {\color{black} $t = 20 \,$s}. 
%

After applying controller (\ref{eq:pendubot_u_hybrydowe}) several conditions had to be imposed on the model.
One of them is a restriction imposed on initial angular velocities in the joints due to constraints of control law \eqref{eq:RE_acrobot_u}.
This situation is not favorable, as there is no any initial kinetic energy in the system at the beginning of the Pendubot movement.
Hence, it is required to provide a~great deal of energy to drive the robot from starting to desired position, and not to fall at the very beginning of the movement. 
Moreover, the robot is not fixed to the ground and the reaction forces are monitored preventing the robot to {\color{black}lift-off}.
%

{\color{black}Additionally, algorithm \eqref{eq:RE_acrobot_u} does not guarantee that trajectory $x(t)$ will not reach singularity.} Thus, the singularity condition should be also verified. Due to the presence of singular points the solver and simulation "hangs" or crash. In order to examine if a current simulation scenario is computed properly a time criterion has been used.  As the time can be treated as the simulation time and a real time of executing the program, one should distinguish them. First of them is a "standard" simulation time, computed by a numerical solver. The other one is the time of program execution (or iterations of the program, to become independent of the computer CPU).
%
In simulations it was assumed that if during 50000 iterations of the program execution no result is obtained (namely the desired configuration is not achieved), such a situation is identified as a crash of the current program. 
\\
As mentioned in Section \ref{sec:respondek} the control signal explodes around equilibria. Thus, the additional imposed restriction is the maximal value of $\gamma$ parameter from Algorithm~(\ref{code:alg_1}).
Authors assumed  $\gamma_d = 0.04$, which was obtained by trail end error method.
\\
The additional problem with execution the algorithm is the need of inverting the decoupling matrix $L_g L_f^2 h$ in Eq.~(\ref{eq:RE_acrobot_u}).
Fig.~\ref{fig:re_pendu_mianownik} depicts a plot of $L_g L_f^2 h$ during one exemplary simulation trial, for Case~I and II, respectively.
To overcome the problem of inverting this matrix, authors use an arbitrarily chosen parameter $\epsilon$, which was added to $L_g L_f^2 h$  when was close to $0$.
Parameter $\epsilon$ makes it possible to enlarge the region of attraction of analyzed algorithm, but was not designed by \cite{Respondek_2019}.
This procedure can be treated as an attempt to improve  the performance of the algorithm.
The invertibility of $L_g L_f^2 h$ was another constraint imposed on the model.
%
\begin{figure}[!h]
	\centering
{\footnotesize a)}\includegraphics[width=0.46\linewidth]{img/M_d_bez_sat}
{\footnotesize b)}\includegraphics[width=0.46\linewidth]{img/M_d_z_sat}
\caption{Decoupling matrix -- a) Case I, b) Case II }
	\label{fig:re_pendu_mianownik}
\end{figure}
% 

To see how the algorithm drives the robot to the reference upright position one of the successful trails from Fig.~\ref{fig:obszar_zbierznosci_stero_hybrid_bez_saturacji} and Fig.~\ref{fig:obszar_zbierznosci_stero_hybrid_z_saturacja}  were chosen.
The common exemplary successful initial condition (for Case~I and II) was chosen as {\color{black} $\theta_{1_0} = 115^\circ$ and $\theta_{2_0} = -50^\circ$}.
%
The obtained angular trajectories are shown in Fig.~\ref{fig:re_pendu_th1th2_case_1} and Fig.~\ref{fig:re_pendu_th1th2_case_2} for the first and second Case, respectively.
%
%
\begin{figure}[!h]
	\centering
	\includegraphics[width=0.85\linewidth]{img/pendubot_th1th2_bez_sat}
	\caption{Angular positions of links -- Case I}
	\label{fig:re_pendu_th1th2_case_1}
\end{figure}
%
%
\begin{figure}[!h]
	\centering
	\includegraphics[width=0.85\linewidth]{img/pendubot_th1th2_z_sat}
	\caption{Angular positions of links -- Case II}
	\label{fig:re_pendu_th1th2_case_2}
\end{figure}
%


A wide range of  robotic application require an energy savings or restrictions on its expenditure  (through e.g. motors finite power). 
As a consequence, the total energetic cost of driving the robot to the upright position was calculated during the simulation.
%
The dynamic effort criterion (\ref{eq:kryterium_energetyczne}) was used as an indicator of energy used by the algorithm. The effort was defined as
%
\begin{equation}
\label{eq:kryterium_energetyczne}
T = 
\int_0^{t_{max}} \tau^2 dt
\end{equation}
%
which is the integration of squares of all join torques over time \cite{human_motion_ksiazka}.
%
Figure~\ref{fig:re_pendu_tau}
% and Fig.~\ref{fig:re_pendu_tau_z_sat}
depicts the motor torque expended during stabilization process for exemplary initial conditions.
%
%
\begin{figure}[!ht]
	\centering
	{\footnotesize a)}\includegraphics[width=0.46\linewidth]{img/tau_bez_sat}
	{\footnotesize b)}\includegraphics[width=0.46\linewidth]{img/tau_z_sat}
	\caption{Motor torque -- a) Case I, b) Case II}
	\label{fig:re_pendu_tau}
\end{figure}
%
%
%\begin{figure}[!ht]
%	\centering
%	\includegraphics[width=0.9\linewidth]{img/tau_z_sat}
%	\caption{Motor torque -- Case II}
%	\label{fig:re_pendu_tau_z_sat}
%\end{figure}
%


The drawings depicting the total torque consumption during simulation are presented in Fig.~\ref{fig:re_pendu_tau_total_bez_sat} and Fig.~\ref{fig:re_pendu_tau_total_sat} for the Case~I and  Case~II, respectively.
%
\begin{figure}[!ht]
	\centering
	\includegraphics[width=0.93\linewidth]{img/RE_pendu_wykres_KE_bez_sat}
	\caption{Energetic cost of successful trials  -- Case I}
	\label{fig:re_pendu_tau_total_bez_sat}
\end{figure}
%
%
\begin{figure}[!ht]
	\centering
	\includegraphics[width=0.93\linewidth]{img/RE_pendu_wykres_KE_z_sat}
	\caption{Energetic cost of successful trials  -- Case II}
	\label{fig:re_pendu_tau_total_sat}
\end{figure}
%
One can observe how the energy is being distributed over analyzed trials.
This observation can be useful to pick a specific trial with respect to energy consumption, to test the algorithm using real robot, for example. 


Figure \ref{fig:re_pendu_reakcja_case} 
%and \ref{fig:re_pendu_reakcja_case_2} 
shows vertical component of ground reaction forces for exemplary successful trial.
One can observe that these values do not cross zero, which means that the robot does not lift-off during its movement.
%
%
\begin{figure}[!h]
\centering
{\footnotesize a)}\includegraphics[width=0.46\linewidth]{img/Fr_bez_sat}
{\footnotesize b)}\includegraphics[width=0.46\linewidth]{img/Fr_z_sat}
\caption{Ground reaction forces -- a) Case I, b) Case II}
\label{fig:re_pendu_reakcja_case}
\end{figure}
%
%
%\begin{figure}[!h]
%	\centering
%	\caption{Ground reaction forces -- Case UI}
%	\label{fig:re_pendu_reakcja_case_2}
%\end{figure}
%%



%--------------------------------------------



% % % % % % % % % % % % % % % % % % % % % % % % % % % % % % % % % %


\section{Conclusion}
\label{sec:konkluzje}

{\color{black}The main objective of this paper is to evaluate properties of the new control approach designed for the Pendubot. Based on the obtained results it is clear that the proposed hybrid controller makes it possible to overcome limitations of the algorithms discussed in \cite{Respondek_2019}, namely  the convergence set can be increased significantly. One can conclude that application of linearization technique with respect to the Pendubot is challenging due to singular points. Hence, this approach can be seen rather as the representation of Pendubot properties than an overall recipe  for stabilizing a double inverted pendulum with one actuation. Basically, the algorithm does not guarantee that trajectories of the closed-loop system do not reach singularities. In order to overcome this issue one could employ tracking of feasible reference trajectory. This concept, however, will explored in future works.}

%One needs to i mpose several constraints in order to reflect the nature of the considered object as good as possible.
%



\bibliographystyle{ifacconf-harvard}
\bibliography{bibliografia}


%\bibliographystyle{IEEEtran}
%\bibliography{mybibfile}







\end{document}
